\section{Theoretical background}
This section provides an overview of the fundamental principles underlying the experience.

	\subsection{Feedback}
		A feedback loop is created when all or some portion of the output is fed back to the input. 
		Negative feedback occurs when the fed-back output signal has a relative phase of $\pi$ with respect to the input signal, 
		while positive feedback occurs when the signals are in phase.
	
	\subsection{Operational amplifier}
		An operational amplifier, often abbreviated as op-amp, is a voltage-controlled voltage generator.
		The op-amp's primary function is to amplify the difference in voltage between its two input terminals, usually labeled as the inverting input (-) and the non-inverting input (+). 
		It has a high input impedance, meaning it draws very little current from the input sources, and a low output impedance, enabling it to drive other circuit components effectively. 
		In its ideal form, an op-amp has infinite voltage gain, however, real-world op-amps have limitations due to factors like power supply voltage, noise, and frequency response. 
		Op-amps are often used in circuits with negative feedback.
	
	\subsection{Virtual ground}
		The concept of a virtual ground is a fundamental aspect of negative feedback operational amplifier (op-amp) circuits. 
		In these circuits, a virtual ground is created at the inverting input terminal of the op-amp, even though it is not physically connected to the ground reference. 
		This virtual ground serves as a reference point for analyzing and designing circuits, allowing simplified calculations and intuitive understanding.
		In fact, negative feedback attempts to drive the inverting input to a voltage that matches the non-inverting input, which can result in $V^+ - V^- = \SI{0}{\volt}$ and, consequently, $I^+ = I^-$.
		An op-amp is a voltage-controlled voltage generator, so it does not absorb current at the input (behaves like an open circuit), hence $I^+ = I^- = \SI{0}{\ampere}$. 

	\subsection{Frequency response}
		The frequency response of a circuit is the quantitative description of how its output responds to its input signal, within a range of different frequencies.
		The frequency response characterizes systems in the frequency domain, just as the impulse response characterizes systems in the time domain. 
		In fact, the frequency response is the Fourier transform of the impulse response. \\
		For a signal $x(t)$, the Fourier transform is defined as: $$ X(\nu) = \int_{-\infty}^{\infty} dt \ x(t) e^{-j 2 \pi \nu t} $$ where $\nu$ is the frequency and $j$ is the imaginary unit. \\\\
		The frequency response of a system is defined as: $$ H = \frac{X_{out}(\nu)}{X_{in}(\nu)} $$ where $X_{in}(\nu)$ and $X_{out}(\nu)$ are the Fourier transforms of the input and output signals, respectively.
		The frequency response $H$ lies in the complex field as it employs a complex extension of Ohm's law to articulate the relationship between voltage and current in a circuit:
		$$V = Z \cdot I$$ \noindent where $Z$ is the complex impedance (measured in $\Omega$). \\
		Consequently, to describe the frequency response of a circuit, it is necessary to consider both the magnitude and the phase of the output signal with respect to the input signal as polar representation of $H$ in the complex plane.

	\newpage
	\subsection{Differentiator circuit}
		The differentiator circuit's functionality is firmly grounded in the mathematical concept of differentiation, a fundamental calculus concept.
		Analogous to how differentiation calculates the rate of change of a mathematical function concerning its independent variable, 
		the differentiator circuit accentuates swift variations in input signal amplitude. \\\\
		The differentiator circuit's design involves the integration of passive and active electronic components:
		capacitors and resistors play pivotal roles in shaping the circuit's frequency response, allowing it to differentiate signal components with different frequencies; 
		while the operational amplifier is employed to achieve high gain and accurate differentiation by ensuring that the inverting input adheres to a voltage level analogous to ground potential. \\
		The schematic diagram of the differentiator circuit is illustrated in Figure \ref{fig:differentiator_circuit}. \\\\
		\begin{figure}[H]
		    \centering
		    \includegraphics[width=0.4\textwidth]{figures/differentiator/circuit.png}
		    \caption{Differentiator circuit schematic.}
		    \label{fig:differentiator_circuit}
		\end{figure}
		\noindent 
		Applying the Kirchhoff's circuit laws at the op-amp's inverting input terminal, which is held at ground potential due to the virtual ground concept, 
		it is possible to establish that the current flowing in the capacitor is equal to the current through the resistor, $i_C(t) = i_R(t), \forall t$, and further more: 
		$$ C \cdot \frac{d}{dt}(V_{in}(t) - V^-) = \frac{V^- - V_{out}(t)}{R} $$
		Remembering that $V^- = \SI{0}{\volt}$, by solving the equation for $V_{out}$, 
		it is shown that the output of the differentiator circuit is proportional to the rate of change of the input signal with respect to time: $$ V_{out}(t) = - RC \cdot \frac{d}{dt}V_{in}(t) $$
		Using Fourier's transform and its property, it is possible to further simplify the first order differential equation to obtain the frequency response of the differentiator circuit: 
		$$ V_{out}(\nu) = - RC \cdot j 2 \pi \nu V_{in}(\nu) $$
		\noindent
		Therefore, the frequency response of the differentiator circuit is directily proportional to the input signal's frequency.
		Since all the constants are real and positive, the frequency response lies in the complex field's negative imaginary axis, it is expected to find a phase of $-\frac{\pi}{2}$.
		
	\newpage
	\subsection{Integrator circuit}
		The integrator circuit can be easily obtained from a differentiator circuit by simply swapping the resistor and the capacitor. 
		The schematic diagram of the circuit is illustrated in Figure \ref{fig:integrator_circuit}. 
		\begin{figure}[H]
		    \centering
		    \includegraphics[width=0.5\textwidth]{figures/integrator/circuit.png}
		    \caption{Integrator circuit schematic.}
		    \label{fig:integrator_circuit}
		\end{figure}
		\noindent
		Similar to the approach taken with the differentiator circuit, Kirchhoff's circuit laws can be employed at the virtual ground node to obtain the equation that characterizes the integrator circuit's behavior: $$ \frac{V_{in}(t)}{R} = - C \cdot \frac{d}{dt}V_{out}(t) \quad \Longrightarrow \quad V_{out}(t) = - \frac{1}{RC} \cdot \int_{\SI{0}{\second}}^t d\tau \ V_{in}(\tau) + V(\SI{0}{\second}) $$
		Notably, the output voltage is proportional to the integral of the input voltage over time, with adjustments related to the time constant and the initial conditions of the circuit. \\
		The term $V(\SI{0}{\second})$ represents the initial condition of the output voltage at $t=\SI{0}{\second}$ and introduces an offset or a foundational voltage level to the integrated output signal. 
		This offset can serve as a calibration factor, facilitating the alignment of the output signal with a desired reference level. \\
		Using Fourier's transform and its property, it is possible to further simplify the integral equation to obtain the frequency response of the circuit: 
		$$ V_{out}(\nu) = \frac{j}{RC \cdot 2 \pi \nu} \cdot V_{in}(\nu) $$
		The integrator circuit's frequency response exhibits an inverse relationship with the input signal's frequency and is situated along the positive imaginary axis within the complex domain.
		Therefore it is expected to measure a phase of $\frac{\pi}{2}$. \\
		The limitations of an ideal integrator can be minimized by adding resistor in parallel with the capacitor, which avoids op-amp going into open loop configuration at low frequencies.
		For instance, the initial measurements demonstrated an unexpected shift, it became imperative to introduce a robust resistor ($R=\SI{0.98}{\mega\ohm}$) to the circuit. 
		This strategic inclusion rectified the observed shifts in measurements. 
