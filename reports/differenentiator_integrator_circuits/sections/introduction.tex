\section{Introduction}
An introduction to the principia behind the experiment.

	\subsection{Differentiator circuit}
		The schematic diagram of the differentiator circuit is illustrated in Figure \ref{fig:differentiator_circuit}. \\\\
		The differentiator circuit's functionality is firmly grounded in the mathematical concept of differentiation, a fundamental calculus concept.
		Analogous to how differentiation calculates the rate of change of a mathematical function concerning its independent variable, 
		the differentiator circuit accentuates swift variations in input signal amplitude. \\\\
		The differentiator circuit's design involves the integration of passive and active electronic components:
		capacitors and resistors play pivotal roles in shaping the circuit's frequency response, allowing it to differentiate signal components with varying frequencies, 
		while the operational amplifier is employed to achieve high gain and accurate differentiation by ensuring that the inverting input adheres to a voltage level analogous to ground potential. \\
		\begin{figure}[H]
		    \centering
		    \includegraphics[width=0.8\textwidth]{figures/differentiator/circuit.png}
		    \caption{Differentiator circuit schematic.}
		    \label{fig:differentiator_circuit}
		\end{figure}
		\noindent 
		Applying the Kirchhoff's circuit laws at the op-amp's inverting input terminal, which is held at ground potential due to the virtual ground concept, 
		it is possible to establish that the current flowing in the capacitor is equal to the current through the resistor, $i_C(t) = i_R(t), \forall t$, and further more: 
		$$ C \cdot \frac{d}{dt}(V_{in}(t) - V^-) = \frac{V^- - V_{out}(t)}{R} $$
		Remembering that $V^- = \SI{0}{\volt}$, by solving the equation for $V_{out}$, 
		it is shown that the output of the differentiator circuit is proportional to the rate of change of the input signal with respect to time: $$ V_{out}(t) = - RC \cdot \frac{d}{dt}V_{in}(t) $$
		
	\subsection{Integrator circuit}
		The integrator circuit can be easily obtained from a differentiator circuit by simply swapping the resistor and the capacitor. 
		The schematic diagram of the circuit is illustrated in Figure \ref{fig:integrator_circuit}. \\\\
		Similar to the approach taken with the differentiator circuit, Kirchhoff's circuit laws can be employed at the virtual ground node to obtain the equation that characterizes the integrator circuit's behavior: $$ \frac{V_{in}(t)}{R} = - C \cdot \frac{d}{dt}V_{out}(t) \quad \Longrightarrow \quad V_{out}(t) = - \frac{1}{RC} \cdot \int_{\SI{0}{\second}}^t V_{in}(\tau)d\tau + V(\SI{0}{\second}) $$
		Notably, the output voltage is proportional to the integral of the input voltage over time, with adjustments related to the time constant and the initial conditions of the circuit. \\
		The term $V(\SI{0}{\second})$ represents the initial condition of the output voltage at $t=\SI{0}{\second}$ and plays a pivotal role in shaping the behavior of the circuit, as it introduces an offset or a foundational voltage level to the integrated output signal. 
		This offset can serve as a calibration factor, facilitating the alignment of the output signal with a desired reference level. \\
		For instance, in cases where the initial measurements demonstrated an unexpected shift, it became imperative to introduce a robust resistor ($R=\SI{0.98}{\mega\ohm}$) to the circuit. 
		This strategic inclusion was necessary to rectify the observed shifts in measurements and to achieve the desired circuit response. 
		The high-value resistor played a crucial role in mitigating these deviations and aligning the measurements with the intended outcomes.
		\begin{figure}[H]
		    \centering
		    \includegraphics[width=0.9\textwidth]{figures/integrator/circuit.png}
		    \caption{integrator circuit schematic.}
		    \label{fig:integrator_circuit}
		\end{figure}
