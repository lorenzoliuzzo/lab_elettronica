% !TeX root = relazione.tex
\documentclass{article}

\usepackage[utf8]{inputenc}
\usepackage[a4paper, total={15.3 cm, 21.3 cm}]{geometry}
\usepackage{amsmath}
\usepackage{amssymb}
\usepackage{gensymb}
\usepackage{booktabs}
\usepackage{hyperref}
\usepackage{caption}
\usepackage{float}
\usepackage{graphicx}
\usepackage{subfig}
\usepackage{titlesec}
\usepackage{titletoc}
\usepackage[dvipsnames]{xcolor}
\usepackage{longtable}
\usepackage{calc}
\usepackage{array}
\usepackage{subfiles} % Best loaded last in the preamble
\usepackage{etoolbox}
\usepackage{xparse}

\hypersetup{colorlinks=true,linkcolor=black}
\renewcommand\thesection{\arabic{section}}
\titlecontents{chapter}[1.05em]{\bigskip}
{\contentslabel[\MakeUppercase{\romannumeral\thecontentslabel}]{1em}\enspace\textsc}
{\hspace*{-1em}\textsc}
{\hfill\contentspage}
\titlecontents{section}[1.6em]{\smallskip}
{\thecontentslabel.\enspace}
{}
{\titlerule*[1pc]{.}\contentspage}
\setcounter{tocdepth}{2}


\begin{document}

    \pagenumbering{roman}
    \thispagestyle{empty}

    \begin{center}

        \includegraphics[width=1.\linewidth]{../logo.jpg}
        \centering
        \vspace{3cm}

        \uppercase{\Large Laboratorio di Elettronica:\\ lezione 01 \par}
        \vspace{3cm}

        \Large Lorenzo Liuzzo \par
        \vspace{1.5cm}

        \Large \today \par

    \end{center}
    \clearpage


    \section{Obiettivi del corso}
        \begin{itemize}
            \item Conoscere i concetti fondamentali dei circuiti elettrici. \par
            \item Misure delle grandezze elettriche sia in continua sia nei domini del tempo e della frequenza. \par
            
        \end{itemize}


    

    


\end{document}